\usepackage[]{mathpazo} %Use option osf for old-style figures and small-caps; use sc for small-caps only.

\linespread{1.1538}         % Palatino needs more leading space (space between lines), and this produces a 15/11 ratio between leading space and font size.  This helps with giving inline math equations more room to breathe, and it balances out the relatively wide lines on the page.

\def\mySfFamily{\sffamily\sansmath} %Needed to turn on sans-serif math fonts locally

\renewcommand{\textsf}[1]{{\sffamily\sansmath#1}} %And a command for sans-serif text


%mathpazo somehow disables the \textsc command, so we need to bring it back.

\renewcommand{\textsc}[1]{\text{\scshape #1}}

\renewcommand{\sc}[1]{\text{\scshape #1}}




%Defining a spaced small caps font.  Needed for the running header, etc.

\newcommand{\textssc}[1]{\text{\scshape \fontfamily{ppl}\selectfont \textls[100]{\MakeLowercase{#1}}}}

\renewcommand{\sfdefault}{uop} %Uses Optima for the sans-serif font

\usepackage[T1]{fontenc}





%Getting old-style numerals

\newcommand{\oldstyle}[1]{\lowercase{\fontfamily{pplj}\selectfont #1}}
